\documentclass{report}
\usepackage{fullpage}
\usepackage{tabu}
\renewcommand{\baselinestretch}{2}
\setlength{\parindent}{4em}
\setlength{\parskip}{1em}

\author{Author : Fahim Khan \\
		Mentor : Nilesh Khandelwal
}


\title{Market Regime Detection using Hidden Markov Model}
\begin{document}
\maketitle
\tableofcontents



\chapter{Objective}
There is always challenge for quantitative trader to find out the frequent behaviour of financial market due to change in government policy,negative news item,regulatory environment and other macroeconomics effects. Such periods are known as Market Regime.
These various regimes lead to adjustments of asset returns via shifts in their means, variances,
autocorrelation and covariances. This impacts the effectiveness of time series methods that rely
on stationarity.
There is a clear need to effectively detect these regimes. This aids optimal deployment of
quantitative trading strategies and tuning the parameters within them.\par

This project is an attempt to find out such market regime and accordingly adjust the strategy. The pricipal method used to detect market regime is known as Hidden Markov Model which is a statistical time series techinique.


\chapter{What is Hidden Markov Model?}
Before knowing about Hidden Markov Model, it important to understand Markov Model. The Markov Model is a stochastic state space model involves random transitions between states where the probability of the jump is only dependent upon the current state, rather than any of the previous states. The model is said to possess the Markov Property and is thus "memoryless".

Markov Models can be categorised into four broad classes depending upon the autonomy
of the system and whether all or part of the information about the system can be observed at
each state.

 
 

\begin{tabu} to 0.8\textwidth { | X[l] | X[c] | X[r] | }
 \hline
  & Fully Observable & Partially Observable \\
 \hline
   Autonomous & Markov Chain  & Hidden Markov Model  \\
\hline
Controlled & Markov Decision Process  & Partially Observable Markov Decision Process \\
\hline
\end{tabu}


If model is both autonomous and fully observable. It cannot be modified by actions of an agent as in the controlled processes and all information is available from the model at any point in time.

If the model is fully autonomous but only partially observable then it is known as a Hidden Markov Model. In such a model there are underlying latent states and probability transitions between them but they are not directly observable. Instead these latent states influence the observations. 

The HMM is more familiar in the speech recognition community and communication systems, but during the last years gained acceptance in finance as well as economics and management science.


\chapter{Implementation details}
\section{Programing Language}
The project has been implemented in Python version 2.7. Python being a open source language is most popular for data analysis as it has wide range of available packages for data analysis,machine learning and statistical analysis. It is very popular language and easy to use. 
\section{Packages}
\section{Strategy}
\section{Data}

\chapter{Backtesting Code}
\chapter{Strategy code without using HMM}
\chapter{Strategy code with using HMM}
\chapter{Findings}
\chapter{Future Work}
\chapter{Reference}

\end{document}
